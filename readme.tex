On-Informational-Nudging-and-Control-of-a-payoff-learning-process
=================================================================
\documentclass[english]{article}
\usepackage[T1]{fontenc}
\usepackage[latin9]{inputenc}
\usepackage{hyperref}
\usepackage{url}
\usepackage{babel}
\begin{document}

\title{README for ``On informational nudging and control of payoff-based learning'' MATLAB source code}


\author{Robin Guers, Cedric Langbort and Dan Work}
\maketitle
\begin{abstract}
This document describes the implementation of the simulation used in ``On informational nudging and control of payoff-based learning'' by R.Guers C. Langbort and D. Work, in the process of being submitted for publication in Automatica.
A preprint of the article is available for download on the second
author's website. The source code is hosted at \url{https://github.com/RobinGuers/On-Informational-Nudging-and-Control-of-a-payoff-learning-process}.
\end{abstract}

\section{License}

This software is licensed under the \emph{University of Illinois/NCSA
Open Source License}:

\begin{center}
Copyright (c) 2013 The Board of Trustees of the University of Illinois.
All rights reserved.
\par\end{center}

\begin{center}
Developed by: Department of Civil and Environmental Engineering University
of Illinois at Urbana-Champaign \url{https://github.com/RobinGuers/On-Informational-Nudging-and-Control-of-a-payoff-learning-process}
\par\end{center}

Permission is hereby granted, free of charge, to any person obtaining
a copy of this software and associated documentation files (the \textquotedbl{}Software\textquotedbl{}),
to deal with the Software without restriction, including without limitation
the rights to use, copy, modify, merge, publish, distribute, sublicense,
and/or sell copies of the Software, and to permit persons to whom
the Software is furnished to do so, subject to the following conditions:
Redistributions of source code must retain the above copyright notice,
this list of conditions and the following disclaimers. Redistributions
in binary form must reproduce the above copyright notice, this list
of conditions and the following disclaimers in the documentation and/or
other materials provided with the distribution. Neither the names
of the Department of Civil and Environmental Engineering, the University
of Illinois at Urbana-Champaign, nor the names of its contributors
may be used to endorse or promote products derived from this Software
without specific prior written permission.

THE SOFTWARE IS PROVIDED \textquotedbl{}AS IS\textquotedbl{}, WITHOUT
WARRANTY OF ANY KIND, EXPRESS OR IMPLIED, INCLUDING BUT NOT LIMITED
TO THE WARRANTIES OF MERCHANTABILITY, FITNESS FOR A PARTICULAR PURPOSE
AND NONINFRINGEMENT. IN NO EVENT SHALL THE CONTRIBUTORS OR COPYRIGHT
HOLDERS BE LIABLE FOR ANY CLAIM, DAMAGES OR OTHER LIABILITY, WHETHER
IN AN ACTION OF CONTRACT, TORT OR OTHERWISE, ARISING FROM, OUT OF
OR IN CONNECTION WITH THE SOFTWARE OR THE USE OR OTHER DEALINGS WITH
THE SOFTWARE.


\section{Publishing results using this software}

We kindly ask any future publications using this software include
a reference to the following publication:

Guers Langbort and Work, ``On informational nudging and control of payoff-based learning''

\section{Structure of the code}
Below we give a description of scripts included in the software.
\begin{itemize}
\item \begin{verbatim}Plotl2R21Diagram.m\end{verbatim}
contains the implementation of the stability analysis presented in subsection 3-2 of the two alternative cases.
The functions
\begin{itemize}
\item \begin{verbatim}R21Critical1.m\end{verbatim} and
\item \begin{verbatim}R21Critical1.m\end{verbatim}
draw the upper and lower boundaries of the unstable region.
\end{itemize}


\item \begin{verbatim} MinimumLies2Alternatives.m\end{verbatim} is used to compute the minimum lies to nudge a user toward a specific equilibrium when the user faces two alternatives. It uses the functions
\begin{itemize}
\item \begin{verbatim}objToMinimize.m\end{verbatim} which returns a value measuring how far from equilibrium point the system is.
\end{itemize}


\item \begin{verbatim}DrawLieRegion.m\end{verbatim} can be used to visualize the admissible lying region for the case with three alternatives. It uses the function
\begin{itemize}
\item \begin{verbatim}plotregion.m \end{verbatim} available on Matlab exchange thanks to  Per Bergström
\end{itemize}



\item \begin{verbatim}MinimumLies3Alternative.m\end{verbatim} is used to compute the minimum set of lies to nudge a user toward a specific equilibrium when the user faces three alternatives.
The functions used in this program are
\begin{itemize}
\item \begin{verbatim}PayoffToProbDeltaX.m\end{verbatim} which converts a payoff into a probability value using the logit rule.
\item \begin{verbatim}objfunDeltaX.m\end{verbatim} which returns a value measuring how far from equilibrium point the system is.
\end{itemize}


\item \begin{verbatim}AdaptiveConstantPayoff.m\end{verbatim} is used to simulate the discrete stochastic process presented in the section AdaptiveLyingStrategy.
The functions used are
\begin{itemize}
\item \begin{verbatim}AdaptiveLieDelta.m \end{verbatim} which computes the appropriate adaptive lie.
\item \begin{verbatim}InterestingQuantities.m \end{verbatim} which computes the quantity described in the proof of the adaptive lying strategy.
\end{itemize}
It also uses the classes
\begin{itemize}
\item \begin{verbatim} Player.m \end{verbatim} which contains the characteristic of a user and her ability to update and pick a choice at every time step.
\item \begin{verbatim} Network.m \end{verbatim} which encompasses more features than actually needed.
\end{itemize}


\end{itemize}
\section{Running the code}
The provided m-files can be used to reproduce the results presented in the publication.
\begin{enumerate}
\item Create the illustration of the stability region for section 3.2 by running the script\begin{verbatim}Plotl2R21Diagram.m\end{verbatim} where $\beta$ and the precision can be modified.

\item Draw the admissible lying region region for the case with three alternatives running the script
\begin{verbatim} DrawLieRegion.m\end{verbatim}.
The values of the logit parameter $\beta$, and the true reward R can be modified.
The default values are $\beta=1$ and R=[1,2,3].

 \item Generate the set of minimum constant lies for the case with two alternatives by running the script
 \begin{verbatim} MinimumLies2Alternatives.m\end{verbatim}.
The value of the logit parameter $\beta$, Pibegin,Piend and Pistep can be modified.
The default values are $\beta=1$, Pibegin=0.05, Piend=0.95 and Pistep=10.
 
 \item Generate the set of minimum lies for a specific true reward for the case with three alternatives by running the script
 \begin{verbatim}MinimumLies3Alternative.m \end{verbatim}.
The values of the logit parameter $\beta$,the true Reward R, PiMax, Pimin, NumPoint can be modified.
The default values are $\beta=1$, R=[1,2,3], PiMax=0.95, Pimin=0.05, NumPoint=5.
 
 \item Generate the simulation of the discrete stochastic process, by running
 \begin{verbatim}AdaptiveConstantPayoff.m\end{verbatim}
 where the true reward (GlobalPayofAv), the number of choice (ChoiceNumber), the number of different type of user (PlayerNumber), the number of step simulated, the desired payoff (xstar), the logit parameter $\beta$, and the averaging factors can be modified (the fast factor is 1/n and the slow one is $1/n^{1/2+\alpha}$. The default values are $\beta=1$, ChoiceNumber=3, PlayerNumber=5, StepNumber=100, xstar=[-1,0,1].
\end{enumerate}
\end{document}
